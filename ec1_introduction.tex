%\vspace{-15em}
\chapter{Introduction}
\thispagestyle{empty} %not enumerate the first page
\label{cha:1_intro}

\section{Overview}
Vector-borne diseases are a main world concern, representing $17\%$ of all infectious diseases and more than 1 million deaths annually. These can be transmitted from humans or animals to humans by the so called vectors. Although there are several species of vectors, the most effective vector is the mosquito that become a host after feeding from a infected person and then spread the disease-causing pathogen to uninfected people. The main population affected are those living in the tropical region. Particularly, dengue and malaria, transmitted by mosquitoes, are the most sensible vector-borne diseases because the former has the highest incidence growth in the last 50 years (30-fold) with a 2.5 billion people at risk and the second has the highest morbidity incidence  with an estimated of 627 thousand deaths in 2012.~\cite{WorldHealthOrganisation2014a}\\

The presence and incidence of vector-borne diseases are determined by the vector-pathogen-host relationship, where the environment and the socio-economic factors play a main role~\cite{Tabachnick2010}. For the dengue case, the climate (temperature and precipitation), population growth, international travel, poverty and lack of sustained programmes are the main factors of incidence~\cite{Guzman2010, Gubler2006}. Similarly for malaria,
climate (temperature, rainfall and relative humidity), sustained programmes, social and economic status are major factors~\cite{Huang2011}. However, due to the different behaviour of the vectors, these common factors could have different influences in the incidence of dengue and malaria. \\


% The climate modifies directly the vector, pathogen and host behaviour influencing, mainly, the life expectancy, density population of the vector~\cite{Tabachnick2010}. In Brazil, for instance, it influences the seasonal component of dengue incidence. On the other hand, socio-economic factors can facilitate the transmissions. It is thought that current incidence increase of vector-borne diseases is due to unplanned population growth and the deterioration of housing leading to a lack of basic services (e.g., water, sewer, and waste management)~\cite{Gubler2010}.\\
  



 %That is the reason why Brazil, in particular the brazilian amazon, is of high interest.


%These diseases are commonly found in tropical and sub-tropical regions and places where access to safe drinking-water and sanitation systems is problematic. That is the reason why Brazil, in particular the brazilian amazon, are of high interest.


%Dengue is a mosquito-borne viral disease that has rapidly spread in all regions of WHO in recent years. Dengue virus is transmitted by female mosquitoes mainly of the species Aedes aegypti and, to a lesser extent, A. albopictus. The disease is widespread throughout the tropics, with local variations in risk influenced by rainfall, temperature and unplanned rapid urbanization. There are 4 distinct, but closely related, serotypes of the virus that cause dengue (DEN-1, DEN-2, DEN-3 and DEN-4). Recovery from infection by one provides lifelong immunity against that particular serotype. However, cross-immunity to the other serotypes after recovery is only partial and temporary. Subsequent infections by other serotypes increase the risk of developing severe dengue.









%Malaria is caused by a parasite called Plasmodium, which is transmitted via the bites of infected mosquitoes. In the human body, the parasites multiply in the liver, and then infect red blood cells. Symptoms of malaria include fever, headache, and vomiting, and usually appear between 10 and 15 days after the mosquito bite. If not treated, malaria can quickly become life-threatening by disrupting the blood supply to vital organs. In many parts of the world, the parasites have developed resistance to a number of malaria medicines. Key interventions to control malaria include: prompt and effective treatment with artemisinin-based combination therapies; use of insecticidal nets by people at risk; and indoor residual spraying with insecticide to control the vector mosquitoes.

Brazil, a tropical country with the highest population in Latin America zone, has favourable climate conditions for mosquito vectors being one of the most affected countries by arboviral diseases. For instance,  it is the country with highest incidence of dengue with peaks from December to May because of climate conditions~\cite{Amancio2015}. Specifically, the geographic, climatic and socio-economic features of the Amazon region make of this a susceptible and endemic area of vector-borne diseases. More than $95\%$ of the viruses responsible of vector-borne diseases of the country has been isolated in the Amazon region~\cite{Paula2015}. In addition, while some diseases has been successfully controlled in the rest of Brazil, they are still a main concern in the Amazon basin; this is the case of the malaria disease~\cite{Achcar2011}.\\

% Most of the cases occur from December to May ~\cite{Amancio2015} because of climate conditions.

%Brazil is the largest and most populated country in Latin America, covering >8 million km2 with an estimated 2002 population of 174,632,932 inhabitants. High popula- tion density areas and cities (up to 12,901 inhabitants/km2) are located mainly on the Atlantic Coast. Most of Brazil has a tropical climate; in the southern region, the climate is subtropical. The rainy season is observed in the first sev- eral months of the year, and the average temperature is >20�C (14).

%Currently, Brazil is the leading country in terms of the number of dengue cases reported. Seasonality with most of the cases ocurring from December to May ~\cite{worldwideAmancio2015}.


% Brazilian Amazon
Actions to control malaria and dengue were mainly focus on the reduction of the vector population and in education of the community about these diseases~\cite{Marzochi1994, Barat2006}. For dengue, in Brazil and the Americas, there was a successful program to eradicate the vector which explain the absence of dengue outbreaks between 1923-1981; however, due to the discontinuance of the program, the vector reinvaded the country becoming a today burden~\cite{Figueiredo1996, Gubler2006}. Similarly, in 1940 satisfactory results were obtained in the campaign against a main malaria vector in North-eastern Brazil and currently a significant reduction of cases has been reported in the country; nevertheless, the same success was not obtained in the Amazon region~\cite{Coura2006}. Despite the efforts to control dengue and malaria, these vector-borne diseases are still affecting the population and their quality of life; even worse, dengue is showing new trends that could aggravate the situation like hyper-endemicity and increased genetic diversity~\cite{Figueiredo2012}.\\

In this context, the prediction of incidence is a need to improve control programs in order to prevent outbreaks with an efficient distribution of logistics and human resources to the affected zones within a reasonable time. Although, several attempts to predict epidemics has been made, the influence of the risk factors, the spatial variation and time evolution are not still fully understood. In part it this due to the complexity of vector-pathogen-host ecosystem with emerging patterns in dengue disease ~\cite{Paula2015}, but the high computationally cost that is required for sophisticated spatio-temporal models for disease mapping and prediction has been a limiting factor.\\

In order to determine the main risk factors affecting malaria and dengue incidence in the Brazilian Amazon between 2006-2013 and to propose a predictive spatio-temporal model, Bayesian hierarchical techniques in the framework of latent Gaussian models were used through the novel INLA (Integrated Nested Laplace Approximation) inference approach~\cite{Rue2009}. The area of study covers 310 municipalities of 6 Federative Units and the considered factors include climatic and socio-economic variables in space, time and space-time domains. An additional goal was also to assess the efficiency of the INLA approach in comparison with Markov Chain Monte Carlo (MCMC) methods for spatial models.\\

This dissertation continues by providing major information about the epidemiology features of dengue and malaria join with a discussion of the literature review of spatio-temporal modelling of dengue and malaria incidence in this chapter. An exploratory analysis of the covariates and an initial analysis of the temporal, spatial and spatio-temporal patterns are presented in chapter \ref{cha:2_intro}. It will then go on to the explanation of the Laplace Approximation and the Bayesian inference in latent Gaussian models through the Integrated Nested Laplace Approximation in chapter \ref{cha:3_intro}. Chapters \ref{cha:4_intro}, \ref{cha:5_intro} and \ref{cha:6_intro} contain the dengue and malaria incidence modelling joint with the comparison of models and diagnostics for the temporal, spatial and spatio-temporal domains respectively. Finally, concluding remarks of the study are presented in chapter \ref{cha:7_intro}.\\

%\section{Epidemiology of Dengue and Malaria}

%\section{Introduction}
%%\section{Vector-borne diseases}
%
%Vector-borne diseases are transmitted by blood-sucking arthropods, mainly insects and ticks, from one person or animal to another. These illness covers more than $17\%$ of all infectious diseases ~\cite{Tabachnick2010} and more than 1 million deaths annually ~\cite{WorldHealthOrganisation2014a}.  
%Malaria and dengue are the most sensible vector-borne diseases because the former has the highest morbidity incidence  with an estimated of 627 000 deaths in 2012 and the second has had the highest incidence grow in the last 50 years ~\cite{WorldHealthOrganisation2014a}. Although there are different factors associated with these diseases, climatic and socio-economic variables are more related with those being more prevalent in tropical and sub-tropical regions and poor areas, probably with absence or low quality of water supply, sewer and waste management systems ~\cite{Gubler2006}. %That is the reason why Brazil, in particular the brazilian amazon, is of high interest.
%
%\subsection{Dengue}
%%These diseases are commonly found in tropical and sub-tropical regions and places where access to safe drinking-water and sanitation systems is problematic. That is the reason why Brazil, in particular the brazilian amazon, are of high interest.
%There are four zerotypes (DENV-1, -2, -3 and -4) for dengue viruses mainly transmitted by the female mosquito of the species \emph{Aedes aegypti} and, with minor efficiency, by the \emph{Aedes albopictus} ~\cite{Gubler2006}. After being bitten, the virus undergoes in an incubation period of 3-14 days and the infection might be asymptomatic, mild dengue fever (DF), dengue haemorrhagic fever (DHF), or dengue shock syndrome (DSS). Symptoms include frontal headache, retro-orbital pain, myalgias, arthralgias, haemorrhagic manifestations, rash, and low white blood cell count ~\cite{Gubler2006}. Some risk factors could be climate, environment, water quality and management, education, air pollution, natural disasters and social ~\cite{Rajeswari2015} 
%
%%Dengue is a mosquito-borne viral disease that has rapidly spread in all regions of WHO in recent years. Dengue virus is transmitted by female mosquitoes mainly of the species Aedes aegypti and, to a lesser extent, A. albopictus. The disease is widespread throughout the tropics, with local variations in risk influenced by rainfall, temperature and unplanned rapid urbanization. There are 4 distinct, but closely related, serotypes of the virus that cause dengue (DEN-1, DEN-2, DEN-3 and DEN-4). Recovery from infection by one provides lifelong immunity against that particular serotype. However, cross-immunity to the other serotypes after recovery is only partial and temporary. Subsequent infections by other serotypes increase the risk of developing severe dengue.
%\subsection{Malaria}
%Malaria is caused by the parasite Plasmodium (\emph{P. falciparum}, \emph{P. vivax}, \emph{P. ovale}, and \emph{P. malariae}) and transmitted by the female \emph{anopheline} mosquitoes. More than 450 Anopheles species are known, 60 of these are considered actual vectors in the wild ~\cite{Cohuet2010}. The symptoms of Malaria can include fever, headache, and vomiting between 10 and 15 days after the infection. The evolution of the disease is also highly dependent on environmental, geography  and socio-economic factors ~\cite{Cohuet2010, Haque2009}.
%
%
%
%
%
%
%
%%Malaria is caused by a parasite called Plasmodium, which is transmitted via the bites of infected mosquitoes. In the human body, the parasites multiply in the liver, and then infect red blood cells. Symptoms of malaria include fever, headache, and vomiting, and usually appear between 10 and 15 days after the mosquito bite. If not treated, malaria can quickly become life-threatening by disrupting the blood supply to vital organs. In many parts of the world, the parasites have developed resistance to a number of malaria medicines. Key interventions to control malaria include: prompt and effective treatment with artemisinin-based combination therapies; use of insecticidal nets by people at risk; and indoor residual spraying with insecticide to control the vector mosquitoes.
%\section{Brazil and its Amazon region}
%Brazil is a tropical country with the highest population in Latin America. Its environmental and socio-economic conditions, particularly the brazilian region, makes of this an endemic region for vector-borne diseases such as dengue and malaria. Currently, it is the country with major dengue cases reported occurring mainly from December to May ~\cite{Amancio2015} because of climate conditions.
%% Most of the cases occur from December to May ~\cite{Amancio2015} because of climate conditions.
%
%%Brazil is the largest and most populated country in Latin America, covering >8 million km2 with an estimated 2002 population of 174,632,932 inhabitants. High popula- tion density areas and cities (up to 12,901 inhabitants/km2) are located mainly on the Atlantic Coast. Most of Brazil has a tropical climate; in the southern region, the climate is subtropical. The rainy season is observed in the first sev- eral months of the year, and the average temperature is >20�C (14).
%
%%Currently, Brazil is the leading country in terms of the number of dengue cases reported. Seasonality with most of the cases ocurring from December to May ~\cite{worldwideAmancio2015}.
%
%
%% Brazilian Amazon
%\subsection{Control and Surveillance of Malaria and Dengue}
%
%Control system to stop the spread of both diseases has been taken through the years. Temporally, dengue was stooped for around 60 years, between 1923 and 1981, in Brazil thanks to
%campaign to eradicate the mosquito ~\cite{Figueiredo1996}. However, the disease returned around 1981 and since the certain approaches of control has been taken: monitoration of mosquito levels, community cleanup programs and direct combat with insecticides ~\cite{Figueiredo1996}. Despite the effort to stop dengue, it is still a main concern and one alternative control is to predict epidemics spatially and prepare the community for the episode with more logistics and human resources. For this reason, the necessity of accurate spatio-temporal models are required.   
%
%Several attempts to predict epidemics has been made, however, the understanding of the influence of the risk factors are not still clear and, in addition, emerging patterns such as severity due to hyperendemicity with increased genetic diversity and increased number of severe cases in children are now present ~\cite{Figueiredo2012, Paula2015}. On the other hand, computational aspects has been also a problem for spatio-temporal models.
%
%\subsection{Objectives}
%
%In this project, bayesian spatio-temporal models,  through the novel INLA (Integrated Nested Laplace Approximation) approach ~\cite{Rue2009}, are used to determine the influence of climatic and socio-economic factor in the incidence of dengue and malaria in 310 municipalities of the brazilian Amazon.
%\subsection{Structure of the Thesis}

%Expect to find

\section{Epidemiology of Dengue and Malaria}

This section covers the main features of dengue and malaria vector-borne diseases; with a focus on the such as the history, vector, pathogen, infection, symptoms and associated factors.

%Although there are different factors associated with these diseases, climatic and socio-economic variables are more related with those being more prevalent in tropical and sub-tropical regions and poor areas, probably with absence or low quality of water supply, sewer and waste management systems ~\cite{Gubler2006}.


\subsection{Dengue: Vector, Pathogen, Symptoms and Risk Factors}

Dengue viruses were initially located in the jungle, then they spread to rural environments probably due to the clearing of the forests for the development of human settlements~\cite{Gubler2006}. All dengue viruses belong to one of the four serotypes (DENV-1, -2, -3 and -4); the infection of one of them provide immunity for that serotype, but not against the others~\cite{Gubler1997}.  It is thought that initially, these viruses were transmitted by peri-domestic mosquitoes such as \emph{Aedes albopictus} and then with major efficiency by the domestic mosquito \emph{Aedes aegypti}~\cite{Gubler2006}. This last prefers to lay its eggs in artificial containers around homes, to rest indoors and feeding during daylight~\cite{Gubler2006}. Female \emph{Aedes aegypti} is a nervous feeder that disrupts its meal with a slight movement to, then, return to the same or a different person being able of infecting several people in just one meal~\cite{Gubler2006}. The infection could  be asymptomatic, mild dengue fever (DF), dengue haemorrhagic fever (DHF), or dengue shock syndrome (DSS); although the last two has major severity, the mechanics to develop severe dengue is not clear~\cite{Wilder-Smith2010}. Symptoms include fever, myalgia, frontal headache, retro-orbital pain, arthralgia, rash, haemorrhagic manifestations and low white blood cell count~\cite{Amancio2015, Gubler1997}. Another symptoms related with a severe case are hypothermia, abdominal pains, rapid breathing and persistent vomiting~\cite{WorldHealthOrganisation2014a}. \\


Between the environmental and socio-economic factors that affect the dengue incidence, the humidity, temperature and precipitation are a strong influencers because they impact in the population density of the \emph{Aedes aegypti} mosquito~\cite{Guzman2010, Murray2013}. On the other hand, the lack of basic resources and the deficiency of low water supply, sewer and waste management systems are of consideration because they are related with the presence of artificial habitat for the incubation of the mosquito eggs~\cite{Gubler2006, Morato2015}. The efficiency of control programs also play an important role, but the level of information and participation of the communities are a key aspect~\cite{Marzochi1994}. In addition, natural disaster such as drought or flood could have a strong impact~\cite{Rajeswari2015}.\\

%Some risk factors could be climate, environment, water quality and management, education, air pollution, natural disasters and social ~\cite{Rajeswari2015} 

\subsection{Malaria: Vector, Pathogen, Symptoms and Risk Factors}

It is thought that species of human malaria parasite has spread from Africa to other continents~\cite{Carter2002}. These parasites belong to four species of the genus Plasmodium (\emph{P. falciparum}, \emph{P. vivax}, \emph{P. ovale}, and \emph{P. malariae}); however, human infectious with \emph{P. knowlesi}, a monkey parasite, has been found in forested areas of South-East Asia~\cite{WorldHealthOrganisation2014}. The main malaria pathogens in Brazil are \emph{P. vivax} and \emph{P. falciparum}~\cite{Carter2002, DeAndrade1995}, which are transmitted mainly by the female \emph{Anopheles darlingi} vector~\cite{Oliveira-Ferreira2010}. It is important to emphasize that this mosquito is found in around the $80\%$ of Brazil, but malaria is almost restricted to the Amazon region covering around the $99.8\%$ of the cases~\cite{Oliveira-Ferreira2010}. It is know, at least out of Africa, that the female mosquito of malaria prefer to feed on animals rather than humans~\cite{Carter2002}, and that shallow groundwater and river edges are common places of breeding for this vector~\cite{CruzMarques1987}. Usually, the symptoms appears between 10 and 15 days after the infection, depending of the \emph{Plasmodium} specie and the partial immunity of the host~\cite{Bruce-Chwatt1971}. They can include fever, chills, rigors, sweating, body aches, headache and nausea; and life-threatening conditions in \emph{P. falciparum} malaria~\cite{Carter2002}. Occurrence of these symptoms in cycles is a high indicator of malaria; nonetheless, this cycle pattern is not common in \emph{P. falciparum} malaria~\cite{Bruce-Chwatt1971}.

Similarly than dengue, environmental and socio-economic conditions are the main risk factors of malaria. Climate variables such as temperature and relativity humidity impact in the biology of the mosquito, whereas rainfall and floodplain areas could increase of reduce possible breeding sites~\cite{CruzMarques1987, Huang2011}. Low levels of education and poverty are also associated with the presence of malaria~\cite{Carter2002, WorldHealthOrganisation2014}. In addition, the efficiency of malaria control programs are a main factor due to the fact that some countries, including Brazil, have obtained satisfactory results against malaria with the available control tools~\cite{Barat2006}. Finally, extreme events influence the development of the vector: for instance, floods could increase the number of breeding sites~\cite{Sewe2015}.\\

\section{Literature Review}


