{\centering \section*{Resumen}}
\singlespace Los organismos bi�ticos y abi�ticos del h�bitat marino erosionan fragmentos de part�culas que se van acumulando en lugares estrat�gicos (ambientes de deposici�n) caracterizados por la estructura del relieve marino y la din�mica existente en este. Los sedimentos finos, �tiles para obtener informaci�n sedimentaria de calidad, generalmente se acumulan en zonas que favorecen la deposici�n continua de sedimentos.  Estos ambientes son caracterizados por la morfolog�a del relieve marino y el comportamiento de ciertos agentes externos como la energ�a hidrodin�mica, los vientos, las corrientes marina, entre otros. El conocimiento de estas zonas brinda aportes significativos para el desarrollo de trabajos geol�gicos \textit{in situ} y como base para estudios de investigaci�n en el Sistema de la Corriente de Humboldt (SCH) del Per�, el cual es el ecosistema marino con mayor cantidad de condiciones extremas e inc�gnitas. Por ello, con el objetivo de predecir la ubicaci�n de probables ambientes de deposici�n continua en el margen continental peruano, se modela espacialmente la presencia de ambientes de deposici�n continua en base a las caracter�sticas de morfolog�a y batimetr�a. En el estudio se obtuvo grillas de alta resoluci�n de batimetr�a  y de probables ambientes de deposici�n continua.  El modelo del kriging log�stico corrobora la asociaci�n entre los ambientes de deposici�n continua y las caracter�sticas morfol�gicas del fondo marino tales como la profundidad, pendientes, lejan�a a las costas entre otras.

\noindent\textbf{\textit{Palabras clave:}} Ambientes de sedimentaci�n continua, Batimetr�a, Correlaci�n Espacial, Geoestad�stica, Morfolog�a, Kriging.\\

{\centering \section*{Abstract}}
Biotic and abiotic organisms in the marine habitat erode particle fragments that are accumulated in strategic locations, called environments of deposition,  characterized by the structure of the topography and the ocean dynamics. Thin sediments, useful  to obtain high-quality sedimentary information, are generally accumulated in areas favoring the continuous deposition of sediments. These environments are characterized by the morphology of the marine topography and the behavior of certain external agents such as hydrodynamic energy, winds, marine currents, and so on. Knowledge of these areas offers significant contributions for the development of \textit{in situ} geological work and it is the basis for research studies in the northern Humboldt Current System (HCS) in Peru, which is the marine ecosystem with the most extreme conditions and unknowns. Therefore, in order to predict the location of continuous deposition environments in Peruvian Continental Margin, the presence of continuous deposition environments are modeled  spatially  based on the morphology and bathymetry characteristics. In the study, high-resolution grids of bathymetry and likely continuous deposition environments were obtained. The logistic  kriging model confirms the association between continuous deposition environments and morphological characteristics of the seabed such as depth, slope, distance to the coast, among others.

\noindent\textbf{\textit{Keywords:}} Bathymetry, Continuous sedimentation environments, Geostatistics, Kriging, Morphology, Spatial Correlation.\\